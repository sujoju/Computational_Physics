\begin{remarkBox}{On How I Implemented \texttt{neural.h}}
I have implemented the energyFunction function to return the corresponding
energy value for a certain spin; this means that I kept the spin as-is, and I did not 
return twice the energy value. 
As for how the energy of a particular spin is calculated, it is using the following:
\begin{equation*}
    E_{ i }
    =
    - \sum_{ j } J_{ i, j } s_{ i } s_{ j }
\end{equation*}
Notice that the sum is over all possible pairs of spins with respect to spin \( i \).
It is in the neuralFlip function that I introduced the factor of \( - 2.0 \) attached
to the energyFunction; this factor of \( - 2.0 \) is what will give 
\( \Delta E_{ \text{flip} } \). 
With this, the implementation of the neuralFlip is almost exactly the same as 
boltzmannFlip from previous assignments and lectures; there are two major differences:

\begin{enumerate}
    \item Flipping the spin now occurs when \( \Delta E_{ \text{flip} } < 0 \) --
        compared to \( \Delta E_{ \text{flip} } \leq 0 \) in the boltzmannFlip method.
    \item I had to include the situation where \( T = 0 \) -- in which, if 
    \( \Delta E_{ \text{flip} } \geq 0 \), then the spin remains unchanged.
\end{enumerate}

As for the getTotalEnergy method, I implemented this method to be exactly the same 
as the getEnergy method from assignment 6. 
In particular, the equation used to calculate the total energy is as follows:
\begin{equation*}
    E
    =
    - \frac{ 1 }{ 2 } \sum_{ i, j } J_{ i, j } s_{ i } s_{ j }
    =
    \frac{ 1 }{ 2 } \sum_{ i } E_{ i }
\end{equation*}
Notice that the factor of \( \frac{ 1 }{ 2 } \) is needed in this case as the energies 
from each spin would be double-counted otherwise. 
Since we have already implemented the method to calculate \( E_{ i } \), all that 
was left to do was to calculate and sum up \( E_{ i } \) for each spin in the lattice 
(while accounting for double-counting).
\end{remarkBox}